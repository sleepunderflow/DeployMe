\chapter{Configuration File}
The injector script is configured by a configuration file \emph{config.conf} located in the same directory as the injector script itself.

Format for the file is as follows:
\begin{lstlisting}
  [Item]
  configuration for item object

  [Command]
  command configuration

  [Item]
  ...

\end{lstlisting}

There is no particular order for command/item entries. The only thing is that they will be added to the binary in the same order as entered to the config file but as they will be separated it doesn't matter whether you first add all the items then commands or command - item - command or however you like.

Configuration options are being added in format key="value"

if value should contain " character it has to be followed by \textbackslash\space(\textbackslash")

\section{Item}
Item settings are in a following format:
\begin{lstlisting}
  [Item]
  name="injected item name"
  path="path to a file to inject"
  permissions="permissions"
  remove-after-use="true/false"
\end{lstlisting}

Path is a relative path to a file to be injected.

Permissions is a desired file permission in a chmod format (for example 755).

Name is the injected item name that can be used internally instead of the ID in the [Command] structure. Has to be a string and can't start with a digit.

If remove-after-use is set to true or missing, the extracted file will be deleted when the client terminates.

\section{Command}
Command settings are in the following format:
\begin{lstlisting}
  [Command]
  command="command to be executed"
  save-output="true/false"
  print-output="true/false"
  condition="condition"
  loop="false/true"
\end{lstlisting}

Command is the desired shell command to execute. It can use the  \textbf{\{ID\}} structure to use the injected items ([Item] part)
If such structure is in the command, the required item will be extracted to a temporary location and \{ID\} will be replaced with the path to the extracted file. The ID is either an integer pointing to which [Item] in order is to be used starting from 0 or a string being the name of the required file ([Item]->name). If the name is used it will be replaced with the index while generating.

If print-output is false the output from the command will not be shown on the terminal. By default it's set to true.

If print-output is true (default) the report from running the command will be generated (\autoref{chapt-Report})

If condition is present, the command will only be run if the condition is true. The condition format is specified in \autoref{chapt-Condition}

If loop is true (implicit default is false) then the command will be repeated until condition becomes false