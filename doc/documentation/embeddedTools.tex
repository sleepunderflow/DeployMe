\chapter{Embedded Tools}
This chapter explains the format for embedded executables for the client.

\section{Format}
Embedded executables and scripts are appended to the end of a client binary by the injector script.

The format of that section is as follows:
\begin{itemize}
  \item Main Header
  \item Item 1 Header
  \item Item 1
  \item Item 2 Header
  \item Item 2
  \item ...
\end{itemize}
\section{Main Header}
The main header of embedded tools is defined as follows:
\begin{lstlisting}[language=C++]
  struct embeddedToolsMainHeader {
    uint32_t totalSize;
    uint16_t numberOfItems;
    char contentHash[32];
  };
\end{lstlisting}

Explanation:

\textbf{totalSize} - the full size of embedded tools in bytes (including main header and individual headers)

\textbf{numberOfItems} - the number of embedded Items

\textbf{contentHash} - the SHA256 hash of the embedded part (Individual headers + items)

\section{Individual headers}

The individual header is separate for each embedded item and placed directly before the item content starts.

It is defied as follows:
\begin{lstlisting}[language=C++]
  struct individualHeader {
    uint16_t ID;
    uint32_t length;
    char fileName[64];
    char itemHash[32];
    uint32_t permissions;
  };
\end{lstlisting}

Explanation:

\textbf{ID} - numeric ID of the item. Starts from 0 and is being incremented by 1.

\textbf{length} - length of the item (excluding individual header)

\textbf{fileName} - the file name of the item including extension 

\textbf{itemHash} - SHA256 hash of the item after unpacking and decrypting

\textbf{permissions} - item permissions to be set by \emph{chmod}

\section{Items}
The item itself is going to be encrypted using one of the individual keys generated and injected using the Injector script. It'll be decrypted when extracting. It'll only be extracted when will have to be used.
