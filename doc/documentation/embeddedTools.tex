\chapter{Embedded Tools}
\label{chapt-embeddedTools}
This chapter explains the format for embedded executables for the client.

\section{Format}
Embedded executables and scripts are appended to the end of a client binary by the injector script.

The format of that section is as follows:
\begin{itemize}
  \item Main Header
  \item Item 1 Header
  \item Item 1
  \item Item 2 Header
  \item Item 2
  \item ...
\end{itemize}
\section{Main Header}
The main header of embedded tools is defined as follows:
\begin{lstlisting}[language=C++]
  struct embeddedToolsMainHeader {
    uint32_t totalSize;
    uint32_t numberOfItems;
    char contentHash[64];
  };
\end{lstlisting}

Explanation:

\textbf{totalSize} - the full size of embedded tools in bytes (including main header and individual headers)

\textbf{numberOfItems} - the number of embedded Items

\textbf{contentHash} - the SHA256 hash of the embedded part (Individual headers + items)

\section{Individual headers}

The individual header is separate for each embedded item and placed directly before the item content starts.

It is defied as follows:
\begin{lstlisting}[language=C++]
  struct individualHeader {
    uint32_t ID;
    uint32_t headerLength;
    uint32_t payloadLength;
    uint32_t flags;
    char fileName[64];
    char itemHash[64];
    char[] additionalData;
  };
\end{lstlisting}

Explanation:

\textbf{ID} - numeric ID of the item. Starts from 0 and is being incremented by 1.

\textbf{headerLength} - size of the individualHeader structure including additional data in bytes

\textbf{payloadLength} - length of the item (excluding individual header)

\textbf{flags} - bitmap that contains settings per item. Currently used bits (bit order in file:|7-0||15-8| - 16-bit integer in little-endian format):
\begin{itemize}
  \item 0: remove-after-use: 0-false, 1-true
  \item 1-31: reserved for future use
\end{itemize}

\textbf{fileName} - the file name of the item including extension 

\textbf{itemHash} - SHA256 hash of the item after unpacking and decrypting

\textbf{additionalData} - extra metadata for the extractor in a form defined in \autoref{sec-additionalData}. The length of that data has to be \texttt{(headerLength-144)} 

\section{Items}
The item itself is going to be encrypted using one of the individual keys generated and injected using the Injector script. It'll be decrypted when extracting. It'll only be extracted when will have to be used.

\section{additionalData}
\label{sec-additionalData}
Any additional metadata and information regarding the individual item is stored here as a continuous string, the format is (there's no spaces in between elements):

\texttt{opcode [length] data}

where opcode is a byte of the given value (in hex) and length parameter is only present for some opcodes.

Currently defined opcodes:
\begin{itemize}
  \item \textbf{0x01 permission} - specify permission for the extracted item, only valid on Linux. Permission is a 3-digit number (ascii characters) representing permissions in chmod format (for example 644 or 777) 
\end{itemize}