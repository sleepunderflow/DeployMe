\chapter{Injected Values}
\label{chapt-injectedValues}

\section{Definition}

There is a structure in the client binary defined as follows:

\begin{lstlisting}
  struct sInjectedConfig {
    uint64_t header               = 0xDEADBEEFDEADBEEF;
    uint64_t flags                = 0x0000000000000000;
    uint64_t injectedDataOffset   = 0x1111111111111111;
  };
\end{lstlisting}

It is statically initialised with those arbitrary values as it is meant to be overwriten by an injector. The header will be detected and then the corresponding values injected.

\section{Fields}

\textbf{header} - A magic value to be detected by an injector. It has to be at the beginning of the structure.

\textbf{flags} - A bitmap for enabling / disabling specific features. Explanation in \autoref{sec-injectedValues-Flags}

\textbf{injectedDataOffset} - The offset in file to the beginning of the Main Header for the Embedded Tools (\autoref{chapt-embeddedTools})

\section{Flags}
\label{sec-injectedValues-Flags}
Currently defined bit flags:
\begin{itemize}
  \item \textbf{bit 0} - \verb|FLAG_EMBEDPRESENT| - if set, Embedded Tools are appended to the file
  \item \textbf{bit 1} - \verb|FLAG_EMBEDENCRYPT| - if set, the encryption of Embedded Tools is enabled. Otherwise just extract
  \item \textbf{bit 2} - \verb|FLAG_ELEVATE| - if set, try to elevate privileges
\end{itemize}

The rest is currently reserved for a future use.